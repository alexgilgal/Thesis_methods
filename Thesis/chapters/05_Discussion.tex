
\part{Discussion}
%Things to discuss:
%- Check the polymorphism in the surface population.
%- Main drivers of the events: ATACseq elements with a lo0t of changes?
%- Why is Astyanax the only species to do this in their caves? See if they are sharing ecosystem with other species. Related to this, which are the mechanisms that explain that. For example, there can be some molecular mechanism that marks regions in the genome and can not be repaired properly. Check the paper of the metilation to see if you can see something in this line. Also check Bilanzija work of the surface that she puts in the dark to see if there is some triggering of the phenotypic plasticity. This will be the base of your discussion, some papers that I have found: \parencite{bilandzija_phenotypic_2020, gore_epigenetic_2018, behrmann-godel_phenotypic_2023, moran_energetic_2015, uller_developmental_2018}
%- Check the GRN of the elements that are under change. This can give insight in which TFs are actually the main drivers of the adaptation. klf genes seems to be important: \parencite{gautam_multi-species_2021}. Also take a look a this paper that Rafa sugested: \parencite{ferrandez-roldan_cardiopharyngeal_2021}
%- Check reparation genes. This is ligated with the previous question, because this could explain the relationship between regulation and mutation. Eg: Methilation is altered and the repairing machinery is not able to repair some regions properly IDK.

%CHECK \parencite{bian_clock1a_2017} FOR THE DISCUSSION.: This may be related to the fact that we find some medodermal changes in cavefish, which has clock Tf alterations. Only bhle40 is affected it appears.

%check: \parencite{yoshizawa_neural_2018} you may want to use Neural crest mess up in the astyanax.
%- Check what is the enhancer that changes in otx2a, and try to find the one that explains the main change in this gene. Look for broken regulatory loops in those enhancers (maybe this would be answered when you do the ananse of the mutated regions). Functional assay???

%- Well check the QTL papers \parencite{warren_chromosome-level_2021} in order to put in context your cavefish results. Use some papers that do not explain well the mechanisms of why the genes are important to use the context of these papers with your results \parencite{leclercq_evolution_2022, ma_hypomorphic_2020}. In the latest one you have an ATAC with SNP and also nearby an accelerated region. Use this joinded result in order to explain the necesity of your data, and how they resolve some questions in the field (this is very nice cabeza).
%- The key thing is that we know many of the mechanisms that drive to the phenotypes that we see from the literature, but what we don't know is why are they happening at the same time, and in a such short amount of time.




\chapter{The effect of GRN connectivity in vertebrate body plan novelties}


Although comparing gene regulation at great evolutionary distances is challenging, it can provide insight into how morphological novelties have arisen. We used amphioxus as a proxy for the chordate ancestor and zebrafish as a vertebrate model to investigate how changes in gene regulation have contributed to the emergence of vertebrate morphological novelties. We have investigated these changes in gene regulation by altering key conserved developmental signalling pathways in amphioxus and zebrafish embryos and then analysing the transcriptomic and epigenomic consequences of these alterations (\ref{sec:Interconnectivity_chapter}). 


Transcriptomic and epigenomic data derived upon pharmacological treatment of amphioxus and zebrafish embryos revealed alterations in several genes expression and accessibility of CREs. Among other affected genes, we found many known effector genes of the treated signalling pathways. In accordance, the biological processes controlled by the affected genes were also under the control of the treated signalling pathways. For example, when disrupting FGF and Wnt pathways, we found that several mesodermal processes were affected, like somitogenesis. Indeed, these signalling pathways are known to specify and control mesodermal development \parencite{kiecker_molecular_2016}. Similar results were obtained for all treatments. When we examined the effect upon Nodal treatment, Nodal antagonists (\textit{lefty} genes) were affected, disturbing the development of mesendodermal structures like the gut. These results are in concordance with the literature, where the role of these signalling pathways in Dorso-Ventral axis patterning in embryos has been extensively described  \parencite{kiecker_molecular_2016, tuazon_temporally_2015}. Moreover, we observed that these processes were affected in both amphioxus and zebrafish, remarking on the evolutionary conservation of these signalling pathways between chordates \parencite{babonis_phylogenetic_2017}. Although the treatments affected similar functions in both species, there were some interesting differences. In zebrafish, the number of genes affected by the treatments was larger than in amphioxus, and they were more related to development. On the contrary, in amphioxus, the affected genes were more closely related to metabolism. This increase in developmental genes responding in zebrafish indicates that, in the transition from invertebrates chordates to vertebrates, these signalling pathways have gained control over more developmental processes. 


Next, we examined the response of orthologous genes between the two species to evaluate whether there is a conserved regulatory logic operating on those. We found that orthologue genes affected similarly by the same treatment tended to be developmental. Overall, we observed that several orthologue genes responded similarly to the same treatment. For example, after RA treatment, the orthologues of RA receptors \textit{rar} genes were overexpressed, altering their known role in central neural system patterning in both species \parencite{bertrand_developmental_2017, kiecker_molecular_2016}. Surprisingly, we found only a small overlap in orthologous genes when analysing Wnt-affected genes. This could suggest a deep rewiring of Wnt in the transition from chordates to vertebrates, which is compatible with the fact that \textit{Wnt} genes have undergone gain and losses of expression domains in the chordate lineage \parencite{somorjai_wnt_2018}. Therefore, we tend to believe that this finding is because, in zebrafish, there has been a notable increase in developmental genes under the control of this pathway.

We then investigated the effect of the treatments at the ATAC-seq level to identify which CREs were under the control of the treated signalling pathways. We identified DARs along the genome of both species responding to the treatments. Interestingly, the genomic distribution of the DARs between the two species is different. In amphioxus, CREs are located closer to genes, whereas zebrafish CREs tend to be in intergenic regions, indicating that gene regulation takes place at greater genomic distances in the latter organism. This phenomenon was first observed when comparing developmental CREs between zebrafish and amphioxus by Marletaz and colleagues \parencite{marletaz_amphioxus_2018}, and our DARs follow the same dynamics since they are indeed a subset of the CREs studied in this work. We also computed TFBS enrichment in DARs in order to know the effector TFs that responded to the treated signalling pathways. Similarly to our RNA-seq analysis, we identified known TF effectors. For example, \textit{tcf} TF is a known effector of Wnt in both vertebrates and amphioxus since it is bound by $\beta$-catenin and enters the nucleus to regulate the transcription of target genes \parencite{bertrand_developmental_2017, steinhart_wnt_2018}. Similarly, when examining TFBS enrichment in DARs that arose due to the overactivation of the RA pathway, we found the nuclear receptor of RAR, \textit{RAR:RXR}, and their secondary target genes, \textit{hox} genes, enriched in these DARs in both species, demonstrating the RA signalling pathway conservation in chordates \parencite{ghyselinck_retinoic_2019}. To illuminate more complex patterns of gene regulation during the treatments, we examined how the genome-wide ATAC-seq signal clustered. We explored the function of the genes nearby these groups of DARs, and we found relatively low similarity scores between zebrafish and amphioxus clusters. This could be because, especially in zebrafish, some of the clusters responded to more than one signalling pathway, potentially affecting more functions and thus reducing the similarity scores. Another possible explanation for these low similarity scores is that the GO enrichment in amphioxus depends on the aforementioned zebrafish orthology groups. By taking the entire orthology group of the amphioxus gene of interest, we may be undermining the GO enrichment analysis in amphioxus due to the inclusion of several genes related to different functions than the original amphioxus gene. However, when using the number of gene families responding to the treatments (Figure \ref{fig:DGS_GOs_fams}), we found no inflation of amphioxus gene families responding to the treatments. We then explored how developmental processes were affected by the treatments. We found that several developmental processes, like neural development, are affected by more than one signalling pathway. Moreover, we observed that, especially in zebrafish, ATAC-seq clusters tend to respond to more than one signalling pathway, indicating that those DARs are under the control of several signalling pathways. By quantifying this observation, we could show that, indeed, zebrafish have more DARs that respond to more than one signalling pathway. Thus, in zebrafish, there is more regulatory information integrated in key developmental genes. Our findings agree with those in \parencite{marletaz_amphioxus_2018}, where it was demonstrated that developmental genes gained regulatory information in the form of CREs during the chordate-to-vertebrate transition. Moreover, our data indicate that this gain of information is also due to the gain of response to key signalling pathways.


Genes that responded at both the transcriptomic and epigenetic level (DSGs) were more abundant in zebrafish than in amphioxus. The reduced number of DSGs in amphioxus could be associated with a loss of regulatory information in this species, as it has been shown that gene losses are also an important driver of evolution \parencite{guijarro-clarke_widespread_2020}. Although we can not rule out this possibility, we are more inclined to think this difference is due to the gain of regulatory input in the chordate-to-vertebrate transition \parencite{marletaz_amphioxus_2018}. In fact, in the work of Guijarro and colleagues, when they analysed the homology groups gained and lost in cephalochordates, they did not find high levels of gaining or losing genes in this group of animals \parencite{guijarro-clarke_widespread_2020}. Considering that, we studied which genes were retained in a 1-to-1 or in a 1-to-many fashion, and we found that the level of connectivity was greater in vertebrates, regardless of these categories of genes. Even so, genes that were retained as 1-to-many showed more connectivity. These genes retained in a 1-to-many manner were characterized by having more CREs and being related to developmental functions \parencite{marletaz_amphioxus_2018}. We think that their higher connectivity reflects those characteristics and that ohnologs retained in a 1-to-many manner have acquired newer CREs, integrating more regulatory information. Adding other species to the analysis helped confirm the gain of regulatory interconnectivity in the invertebrate-to-vertebrate transition, although the experimental setups were not identical. We tested if the gain of regulatory information was due to the acquisition of novel TFBS in conserved CREs, but we could not find evidence for this. Instead, we found that the newer the CREs, the more they integrated more than one signalling pathway, indicating that the gain of regulation is through new CREs that have emerged from WGDs. This fact can help to understand why specialized genes after WGD tend to have more CREs, as was shown in \parencite{marletaz_amphioxus_2018}. In fact, many specialized genes retain only one of the ancestral expression patterns, though they are the genes that have gained more CREs after WGDs. We think that these genes are integrating several signalling and regulatory cues through these new CREs.


To understand the contribution of this increase of GRN connectivity to vertebrate morphological novelties, we used previously available scRNA-seq developmental atlas of zebrafish \parencite{farnsworth_single-cell_2020}. This data allowed us to trace the expression of highly connected genes in specific tissues. We hypothesised that highly connected genes have contributed to the increased tissue complexity needed in vertebrate morphological novelties, like paired appendages or sensory placodes. We find that, in novel tissues, highly connected genes were more expressed that lowly connected genes, reinforcing our hypothesis. It is plausible that GRNs, required in these vertebrate novelties, have been enriched in these genes. For example, in the neural crest, the gene \textit{Pax3/7} is a key gene in the GRNs that control its specification and migration \parencite{simoes-costa_establishing_2015}. In our analyses, this gene in amphioxus is lowly connected, but zebrafish orthologs have gained connectivity, and some of those integrate information from up to three different signalling pathways. Together with the work of Marletaz et al., \parencite{marletaz_amphioxus_2018}, we have demonstrated that WGDs have contributed to the generation of genes that have gained more CREs in vertebrates, likely by restricting and simultaneously specifying the expression domains of these genes. In addition, we have also observed that these new CREs integrate more regulatory information from signalling pathways. We believe that these two phenomena, expression domain restriction and gain of connectivity, have contributed to generating the necessary tissue complexity present in vertebrate morphological novelties.




\chapter{The role of gene regulation to cavefish adaptation}

In this part of this project, we have analysed the role of gene regulation in adapting to a new environment. We have used \textit{Astyanax mexicanus} as a model organism to study adaptation since it presents two different populations, the surfacefish and the cave-adapted morphotype, known as cavefish. The cavefish population presents a wide range of specific phenotypes like loss of pigmentation, loss of the eyes paired with enhanced non-visual sensory capabilities, metabolic changes, circadian rhythm and behaviour alterations \parencite{jeffery_astyanax_2020, oliva_characterizing_2022}. These phenotypes have evolved independently in cavefishes at least in two independent time periods \parencite{fumey_evidence_2018, herman_role_2018, moran_selection-driven_2023}. Accordingly,  \textit{Astyanax mexicanus} constitutes a great model for EvoDevo research, since it allows us to answer many long-standing questions in Evolutionary Biology. Understanding how \textit{Astyanax mexicanus} has adapted to the cave environment expands our knowledge of the molecular, morphological, and physiological mechanisms taking place in the adaptation to this unique environment. In this study, we used \textit{Astyanax mexicanus} to decipher the developmental and gene regulatory mechanisms in cave adaptation. 

We have performed several transcriptomics and epigenomics analyses to compare the GRNs that are different between surfacefish and cavefish. When comparing the ATAC-seq clusters between cavefish and surfacefish, we found several TFBS that were different between the two populations. Then, by integrating ATAC-seq and RNA-seq in the double-selected genes (DSG) and in combination with using ANANSE, there were a series of TFs that consistently appeared as differential between the two populations. We found that the \textit{kruppel-like factors} (klf) transcription factor family is present in several of our differential analyses. These TF factors have several functions in developing key embryonic structures, which are also important for cave adaption. For example, \textit{klf4} TF is enriched in differential ATAC-seq peaks between the two populations. When integrating ATAC and RNA-seq, we find that these TFs were also enriched in the CREs that regulate DSGs, indicating that they are important for cave adaptation GRNs. This TF is an adipogenic factor, and its over-expression induces the differentiation of preadipocytes to adipocytes \parencite{li_kruppel-like_2022}. In our analysis, the binding site of this TF is enriched in those ATAC-seq peaks that are more open at the 24hpf stage, suggesting that the increase in adipose tissue in cavefish could be due to the increase in activity of this TF, among other members of the klf family, during development. Our results pinpoint other klfs with metabolic roles involved in the developmental changes that shaped the cavefish adaption. In \parencite{xiong_early_2018}, Xiong and colleagues find that in \textit{Pachón} cavefishes, the adipose tissue develops early. This early development of fat tissue contributes to the accumulation of fatty tissue, increasing the survivability of these fishes in the cave environment. Our results suggest that the early activity of \textit{klf4} TF can explain this phenotype in \textit{Pachón} cavefish. It is plausible that additional TFs contribute to the striking metabolic differences between cavefish and surfacefish. For example, \textit{hnf4a} is a key TF that drives the GRN changes between surfacefish and cavefish. Our results agree with other studies made in \textit{Astyanax mexicanus} liver cells, which find that the targets of this TF also have differential expression between cavefish and surfacefish livers \parencite{krishnan_genome-wide_2022}. Taken together, our results and the results of Krishnan and colleagues indicate that  the GRN under the control of \textit{hnf4a} is altered, potentially generating the metabolic phenotypes of cavefish. A possible explanation for this is a change in \textit{trans} in the GRN, meaning that \textit{hnf4a} suffers changes in its expression or gene sequence, impairing its function to regulate the target genes. Indeed, this gene has been demonstrated to harbour a potentially crippling mutation \parencite{warren_chromosome-level_2021}. Krishnan and colleagues do not find gene expression changes in \textit{hnf4a} in the liver cells of \textit{Astyanax mexicanus}. In contrast, our results indicate that there are changes in the expression of this gene that could explain this GRN alteration. These contrasting results between our dataset and Krishnan and colleagues are probably due to the different tissue used for the assays, since we used whole embryo samples. Additional metabolism-related TFs with differential activity between the two populations, like the \textit{NFY}, a TF related to lipid metabolism \parencite{lu_nuclear_2015}, confirm previous findings in cavefish and surfacefish adult liver cells (Krishnan et al.). Our results help to understand how the metabolic changes in the cavefish adaptation appear during development. In summary, we demonstrate that the GRNs that control the development of the liver and other metabolism-controlling tissues, like adipocytes, are significantly altered in cavefish while adapting to the cave environment.


Besides the changes related to metabolism, in our analyses, we also detected components of the circadian rhythm, like \textit{clock} and \textit{E-Box} TFs. Cavefish present several phenotypes related to the circadian rhythm, like disrupted sleep and behavioural changes \parencite{beale_circadian_2013, moran_eyeless_2014, yoshizawa_distinct_2015}. In \parencite{mack_repeated_2021}, researchers have found that there are core elements of the circadian rhythm in cavefish which do not show the normal oscillatory transcription, like \textit{cry1a}, \textit{nptx2a} or \textit{nfil3}. However, there are also secondary components of the circadian clock, like \textit{dbpb}, that show normal oscillatory behaviour. In our DSG analysis, we were able to find all the genes related to the circadian rhythm, even the ones that were suggested to show no alterations in their oscillatory behaviour. Moreover, when we analyzed which TFs strongly influenced the adaptation to the cave environment, we identified some elements of the circadian rhythm circuit, like \textit{bhlhe40}, which lost its rhythmic transcription in cavefish. Our results can help to explain the findings of Mack and colleagues' work \parencite{mack_repeated_2021}, where they identified and studied key circadian rhythm components. They could not find sequence differences in the promoter regions of these circadian genes and thus could not explain the differences in the regulatory logic of these genes. Our results complement their findings since our ATAC-seq data can reveal the different activity of distal CREs to these genes. Indeed, several CREs outside the promoter regions of these genes change their activity when comparing surfacefish to cavefish, according to our data. Further exploration of these CREs can give insight into the mechanisms and the key players of circadian rhythm disruption in cavefish adaptation. Additionally, the disruption of cavefish circadian rhythm can lead to several phenotypes which are under the control of the biological clock. In fact, alterations in key components of the circadian rhythm like \textit{bhlhe40} are found in cavefish, where muscle development is partially impaired \parencite{bian_clock1a_2017}. Indeed, \textit{myod}, among other muscle development TFs, is one of the most influential TFs driving cavefish adaptation, according to our data. Our results can provide a potential explanation for why muscle development in cavefish tends to be more irregular in terms of timing compared to surfacefish \parencite{hinaux_developmental_2011}. We believe that the disruption of several components of the circadian rhythm in cavefish alters the muscle development of cavefish, probably generating small differences in the segmentation clock necessary to generate the somites and, thus, making their tempo more irregular between embryos in cavefish. Moreover, not only is muscle development altered in cavefish, but also adult muscle phenotypes have been reported in this population. Olsen and colleagues have studied how the disruption of the circadian clock has affected the muscle in cavefish \parencite{olsen_circadian_2023}. To do so, the researchers have addressed which genes followed a rhythmic transcription in cavefish and surfacefish generating muscle RNA-seq datasets. This approach differs from the previous study \parencite{mack_repeated_2021}, where whole-body RNA-seq was performed. Our results agree with both studies since our DSG and ANANSE analyses reveal the same as them to be altered in the cavefish compared to the surfacefish. All these results show how the adaptation to the cave environment, an environment that lacks circadian cues like daylight, has impacted circadian rhythm components, thus affecting dependent processes important for cavefish adaptation.



One of the most important phenotypes in cavefish is the loss of the eyes. Accordingly, genes related to eye development are downregulated at the 24hpf stage in our RNA-seq analysis. Furthermore, the DSGs of that same stage are also related to visual perception processes. In cavefish, the degeneration process starts at around 24hpf, when first the lens gets apoptotic, and with it, the entire eye structure \parencite{strickler_lens_2007, krishnan_cavefish_2017, jeffery_astyanax_2020, devos_eye_2021}.  
Integrating ATAC-seq and RNA-seq information using ANANSE provided us with the most influential TFs behind this phenotype. Many of the TFs that are central for cavefish adaptation are those related to the eye development GRN, like \textit{vsx2}, \textit{sox}, \textit{irx7} or \textit{otx2a} genes. Our ANANSE analysis, at 24hpf, showed that the most influential TF between cavefish and surfacefish is \textit{vsx2}, which plays an important role in retinal and eye development \parencite{buono_analysis_2021}. According to Buono et al. and Letelier et al., the eye developmental GRN is robust due to the redundancy of TF operating through the same CREs \parencite{buono_analysis_2021, letelier_mutation_2023}. In the latter study, they showed how mutating \textit{vsx} paralogs in zebrafish does not lead to microphthalmia phenotypes, while the same approach in medaka fish and mice does. This data indicated the robustness of the network in zebrafish and the different importance that this TF has across vertebrates. Given that \textit{vsx2} is one of the most influential TF in transitioning from surfacefish to cavefish GRNs, we believe this factor is crucial for eye development in \textit{Astyanax mexicanus}. Another possible explanation, which could complement our previous hypothesis, is that CREs that are bound by \textit{vsx2} and other TFs have suffered mutations in their sequence, impairing the binding of these TFs.
An intriguing observation was that some TFs related to eye development, like \textit{otx2a}, appeared differential at RNA-seq level already at 10hpf. A likely explanation for this is that the specification of the eye field is already altered at earlier stages. In fact, altered spatiotemporal \textit{shh} expression in the ventral midline at 10hpf cavefish embryos leads to an earlier \textit{fgf8} expression, affecting the expression of downstream targets and ultimately, leading to eye morphogenetic defects \parencite{pottin_restoring_2011}. We think that these differences in early eye specification are the ones that make the downstream elements of the eye developmental GRN, like \textit{otx2a}, to be differentiated at so early stages. It is curious though, why the eye still develops until a certain point in development. The research group of Sylvie Rétaux has demonstrated that the cavefish eye normally develops in the early stages of development until the morphogenetic movements sort the telencephalic and hypothalamic cells from the optic cells \parencite{pottin_restoring_2011, devos_eye_2021}. Once the telencephalon and hypothalamus cells are separated from the optic cells, the eye degenerates without affecting other critical neural tissues. This suggests that there are developmental constraints as to when the eye can degenerate, and this is why a very "expensive" and seemingly unnecessary process still takes place in cavefish.


To investigate whether and how changes in the sequence of CREs have influenced cavefish adaptation, we used two approaches. First, we examined CREs with a single nucleotide change in cavefish in comparison to the surfacefish population to assess which TFBSs were affected by these disruptions. The most affected TFBSs were the \textit{klf} gene family, \textit{NFY}, and other TFs known as critical for cave adaptation. A similar approach applied in \textit{Astyanax mexicanus} liver cells \parencite{krishnan_genome-wide_2022} revealed CREs that are responsible for the metabolic phenotype of cavefish. Further exploration of our data also confirmed that these altered CREs were placed near genes with a high impact on cave adaptation processes, as, for example, important for retina specification genes \parencite{buono_analysis_2021}, \textit{hmx1}, \textit{sox11b} and \textit{tead3b}.

Second, we used comparative genomics to understand which CREs are under accelerated evolution in cavefish. These analyses revealed which CREs have undergone DNA changes likely to be explained by more evolutionary pressure.  Several regulatory regions suffered an accelerated evolutionary rate. Most of them are nearby genes that perform key functions in cave adaptation processes, such as olfactory placode development, neural development and several others. These processes are directly related to the different behaviour and metabolism of cavefish compared to surfacefish. Moreover, when we analysed the accelerated CREs that were present in the entire \textit{Astyanax mexicanus} species compared to other teleost species, we found that these accelerated CREs were nearby genes that could potentially help to adapt to the cave environment. This means that in the speciation event of the entire species, some variations could make this species more fit to adapt to the cave environment. A possible explanation for this is that there was some level of standing variation in the genome of the surfacefish that colonized the caves. In agreement with our results, Moran and colleagues have recently demonstrated that this mechanism has largely contributed to the parallel evolution of cave traits in \textit{Astyanax mexicanus} \parencite{moran_selection-driven_2023}. In this work, the genome of several hundreds of \textit{Astyanax mexicanus} individuals was sequenced to find what are the main drivers of parallel evolution in the genome of this fish. They found that most signatures of selection are due to the selection of standing variation in the surfacefish genome that colonized the cave. They also found that \textit{de novo} mutation signatures account for 40\% of the found signatures. In accordance, our results show that accelerated regions are present not only in the cavefish genome but also in surfacefish. Moreover, we found the accelerated selection in CREs regulating important genes for cave adaptation. Also, both studies (Moran et al. and the here-presented study) underline that most signatures for parallel selection are located in non-coding genome sequences. Although the role of \textit{de novo} mutations in the parallel evolution in \textit{Astyanax mexicanus} adaptation to the cave should also be taken into consideration, these results together suggest that the standing variation in ancestral surfacefish has been fixed in the populations that colonized the caves. These mechanisms could explain the fast adaptation of surfacefish to the cave environment \parencite{herman_role_2018}. This is a phenomenon also observed in other species, like the stickleback, in which the selection of standing variation in the populations drives the rapid adaptation to new environments \parencite{jones_genomic_2012, bassham_repeated_2018, reid_threespine_2021}.


Besides this hypothesis, other mechanisms may play a role in cavefish adaptation. We believe that developmental ATAC-seq and RNA-seq experiments are of great importance in resolving current questions in the cavefish community. For instance, many of the detected DARs are within known QTL regions related to cavefish phenotypes such as eye size and metabolism \parencite{mcgaugh_cavefish_2014, warren_chromosome-level_2021}. These modified CREs may explain the regulatory alterations necessary for cave adaptation. Although the lack of functional information limited us, the results of our study can shed light on the findings of several other studies that lack this regulatory information. For example, Leclercq and colleagues \parencite{leclercq_evolution_2022} carried out transcriptomic analyses in the early stages of development in cavefish, surfacefish and F1 hybrids of these two populations. Using this approach, they could determine which genes present variations between cavefish and surfacefish. Due to the inclusion of cavefish and surfacefish F1 hybrids in their analyses, they could determine whether the difference in gene expression was due to changes in CREs. They could do this by computing how biased was the allelic ratio of expression in these F1 hybrids. They found that \textit{rx3}, an important gene that specifies the eye field, has a differential expression due to changes in a nearby CRE. In our data, we could detect an element located 79Kb away from the promoter of this gene, which is not active in the developmental stages we studied. Nevertheless, this does not exclude the possibility of being activated at a different developmental time. We further found that this CRE is bound by TFs, for example, \textit{NFY} and \textit{bhlhe40}, which have already been demonstrated in our study to be important for cavefish adaptation. According to our datasets, this element is a good candidate for causing the differential expression of this gene described by Leclerq et al. Similarly, the \textit{otx2a} is surrounded by CREs that are differentially accessible, some of which harbour SNPs and accelerated regions. Furthermore, our data corroborate those from other studies, such as those published by Ma et al. In this study, \textit{cbsa} gene has defective expression due to changes in a specific CRE of this gene in cavefish \parencite{ma_hypomorphic_2020}. Our data showed changes in the activity of this CRE and additional CREs with differential chromatin accessibility, together with accelerated regions. Our integrative analysis constitutes a great resource for defining all the genes, associated CREs and TFs related to the cavefish specific characteristics. However, it is still one of the many steps to disentangle the cause-and-consequence regulatory cascade that has led to cavefish adaptation. 


In the future, these analyses will be key to deciphering the roles of other mechanisms in cavefish adaptation. Especially interesting to us is the phenotypic plasticity since it has been shown that surfacefish embryos raised in dark conditions can transcriptionally adapt to the new environment \parencite{bilandzija_phenotypic_2020}. The dark-raised fish also have phenotypes that resemble those of cave adaptation, like the presence of smaller eyes. The role of phenotypic plasticity in adaptation to the cave environment can also explain why \textit{Astyanax mexicanus} has been successful in colonising caves in such a short period of time. We hypothesise that this was the first line of adaptation and that as generations in the dark were growing, phenotypic plasticity evolved itself, allowing it to generate the necessary diversity that natural selection needs. As it has been recently demonstrated, other mechanisms play an important role in the genomic adaptation to the new environment, like the selection of standing variation, \textit{de novo}  mutations, but also hybridization between surfacefish that invade the caves and cavefish already present in the cave \parencite{herman_role_2018, moran_selection-driven_2023}. To test thoroughly the role of phenotypic adaptation in cavefish, it would be necessary to perform ATAC-seq and RNA-seq in dark-raised embryos and compare them to our datasets. This comparison will potentially show to what extent phenotypic plasticity plays a role in cave adaptation. 

In summary, our results have shown how gene regulation in cavefish contributes to adaptation to the cave environment. Our setup was based on whole embryo experiments, but the extent of these changes can be further explored by performing single-cell experiments. These experiments would help to determine at a greater resolution which are the GRNs that are behind all the processes of cavefish adaptation at the single-cell level. We could better understand the regulatory changes and how they ultimately affect the whole embryonic tissues.  Finally, the inclusion of other species of fishes that have undergone the same processes as \textit{Astyanax mexicanus}, like the recently discovered \textit{Barbatula barbatula} \parencite{behrmann-godel_phenotypic_2023}, will help to decipher which are the gene regulatory mechanisms that lead to this convergent evolution and which can explain phenotypic plasticity.


Our exploration of the role of gene regulation in different evolutionary processes during this thesis has revealed some important mechanisms. In the first part of this thesis, we have reinforced the idea that WGDs were key in the transition from invertebrate chordates to vertebrates. We went one step further than our previous studies. We demonstrated that not only the gain of regulatory information is important but also that how these new CREs integrate information in key developmental genes is of key importance to generate morphological novelties. In the second part of this thesis, we have revealed the impact of changes in gene regulation in the adaption to a new environment. We have demonstrated how the gene regulation of key developmental genes is altered in surfacefish compared to cavefish. Moreover, we have explored in detail the changes in CREs, and we have observed the direct impact of these changes in key developmental GRNs. This thesis adds to the great body of knowledge of gene regulation and evolution, remarking on the importance of understanding the gene regulatory mechanisms behind evolutionary novelty and adaptation.





