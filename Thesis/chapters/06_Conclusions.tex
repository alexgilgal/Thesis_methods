\chapter{Conclusions}

\begin{enumerate}
    \item The disruption of conserved developmental signalling pathways and the epigenetic analyses of these alterations reveal core genes with differential regulation.
    \item These core genes have gained response to more regulatory inputs provided by conserved signalling pathways. This increase in response in these genes is translated into a gain of interconnectivity between signalling pathways in vertebrates.
    \item The WGDs of vertebrates contributed to this gain in interconnectivity since some copies of vertebrate developmental genes gained newer CREs that were able to integrate more regulatory information.
    \item The comparison of \textit{Astyanax mexicanus} populations during development reveals the modification of several developmental processes related to cave adaptation. These alterations are related to the differential regulation of key developmental genes in the cavefish population.
    \item We identified several alterations in CREs that correlate with the troglomorphic phenotypes of cavefish. We determined the implications of these changes and how they influence cave adaptation processes.
    \item We could identify CREs that change more rapidly than expected, remarking the importance of how strong selective pressure from the environment impacts gene regulation in the adaptation to this new environment.
\end{enumerate}